\PassOptionsToPackage{unicode=true}{hyperref} % options for packages loaded elsewhere
\PassOptionsToPackage{hyphens}{url}
\documentclass[9pt,ignorenonframetext,]{beamer}
\IfFileExists{pgfpages.sty}{\usepackage{pgfpages}}{}
\setbeamertemplate{caption}[numbered]
\setbeamertemplate{caption label separator}{: }
\setbeamercolor{caption name}{fg=normal text.fg}
\beamertemplatenavigationsymbolsempty
\usepackage{lmodern}
\usepackage{amssymb,amsmath}
\usepackage{ifxetex,ifluatex}
\usepackage{fixltx2e} % provides \textsubscript
\ifnum 0\ifxetex 1\fi\ifluatex 1\fi=0 % if pdftex
  \usepackage[T1]{fontenc}
  \usepackage[utf8]{inputenc}
\else % if luatex or xelatex
  \ifxetex
    \usepackage{mathspec}
  \else
    \usepackage{fontspec}
\fi
\defaultfontfeatures{Ligatures=TeX,Scale=MatchLowercase}







\fi

  \usetheme[]{metropolis}






% use upquote if available, for straight quotes in verbatim environments
\IfFileExists{upquote.sty}{\usepackage{upquote}}{}
% use microtype if available
\IfFileExists{microtype.sty}{%
  \usepackage{microtype}
  \UseMicrotypeSet[protrusion]{basicmath} % disable protrusion for tt fonts
}{}


\newif\ifbibliography
  \usepackage[style=abnt,]{biblatex}
      \addbibresource{citations.bib}
  

\hypersetup{
      pdftitle={Autorização de Obras},
        pdfauthor={Luiz Fernando Palin Droubi},
          pdfborder={0 0 0},
    breaklinks=true}
%\urlstyle{same}  % Use monospace font for urls







% Prevent slide breaks in the middle of a paragraph:
\widowpenalties 1 10000
\raggedbottom

  \AtBeginPart{
    \let\insertpartnumber\relax
    \let\partname\relax
    \frame{\partpage}
  }
  \AtBeginSection{
    \ifbibliography
    \else
      \let\insertsectionnumber\relax
      \let\sectionname\relax
      \frame{\sectionpage}
    \fi
  }
  \AtBeginSubsection{
    \let\insertsubsectionnumber\relax
    \let\subsectionname\relax
    \frame{\subsectionpage}
  }



\setlength{\parindent}{0pt}
\setlength{\parskip}{6pt plus 2pt minus 1pt}
\setlength{\emergencystretch}{3em}  % prevent overfull lines
\providecommand{\tightlist}{%
  \setlength{\itemsep}{0pt}\setlength{\parskip}{0pt}}

  \setcounter{secnumdepth}{0}


  \usepackage[brazil]{babel}
  \usepackage{csquotes}

  \title[]{Autorização de Obras}

  \subtitle{em imóveis da União}

  \author[
        Luiz Fernando Palin Droubi
    ]{Luiz Fernando Palin Droubi}

  \institute[
    ]{
    Superintendênia do Patrimônio da União em Santa Catarina
    }

\date[
      \today
  ]{
      \today
        }

\begin{document}

% Hide progress bar and footline on titlepage
  \begin{frame}[plain]
  \titlepage
  \end{frame}


  \begin{frame}
  \tableofcontents[hideallsubsections]
  \end{frame}

\hypertarget{previsuxe3o-legal}{%
\section{Previsão Legal}\label{previsuxe3o-legal}}

\begin{frame}{Previsão Legal}

\begin{itemize}[<+->]
\tightlist
\item
  \alert<1>{A necessidade de Autorização de Obras em imóveis da União está
  prevista no artigo 6º do DECRETO-LEI Nº 2.398\footcite{brasil_dl_1987}, DE 21 DE
  DEZEMBRO DE 1987, cuja redação final foi dada pela LEI Nº
  13.139\footcite{brasil_l13139}, DE 26 DE JUNHO DE 2015}
\end{itemize}

\begin{itemize}[<+->]
\tightlist
\item
  \alert<2>{\emph{Art. 6º Considera-se infração administrativa contra o
  patrimônio da União toda ação ou omissão que viole o adequado uso, gozo,
  disposição, proteção, manutenção e conservação dos imóveis da União.}}
\end{itemize}

\begin{itemize}[<+->]
\tightlist
\item
  \alert<3>{\emph{§ 1º Incorre em infração administrativa aquele que realizar
  aterro, construção, obra, cercas ou outras benfeitorias, desmatar ou instalar
  equipamentos, sem prévia autorização ou em desacordo com aquela concedida, em
  bens de uso comum do povo, especiais ou dominiais, com destinação específica
  fixada por lei ou ato administrativo.}}
\end{itemize}

\end{frame}

\hypertarget{atuauxe7uxe3o-da-spu}{%
\section{Atuação da SPU}\label{atuauxe7uxe3o-da-spu}}

\begin{frame}{Atuação da SPU}

\begin{itemize}[<+->]
\tightlist
\item
  \alert<1>{A atuação da SPU está prevista no §7º do art. 6º}
\end{itemize}

\begin{itemize}[<+->]
\tightlist
\item
  \alert<2>{\emph{§ 7º Verificada a ocorrência de infração, a
  \textbf{Secretaria do Patrimônio da União} do Ministério do Planejamento,
  Orçamento e Gestão aplicará multa e notificará o embargo da obra, quando
  cabível, intimando o responsável para, no prazo de 30 (trinta) dias, comprovar a
  regularidade da obra ou promover sua regularização.}}
\end{itemize}

\begin{itemize}[<+->]
\tightlist
\item
  \alert<3>{Está prevista a aplicação do instrumento judicial de reintegração 
  de posse, em caso de descumprimento}
\end{itemize}

\begin{itemize}[<+->]
\tightlist
\item
  \alert<4>{\emph{§ 11.  Após a notificação para desocupar o imóvel, a
  \textbf{Superintendência do Patrimônio da União} verificará o atendimento da
  notificação e, em caso de desatendimento, ingressará com pedido judicial de
  \textbf{reintegração de posse} no prazo de 60 (sessenta) dias.}}
\end{itemize}

\end{frame}

\begin{frame}{Sanções previstas}
\protect\hypertarget{sanuxe7uxf5es-previstas}{}

\begin{itemize}[<+->]
\tightlist
\item
  \alert<1>{O artigo 6º do DL  prevê a aplicação das seguintes sanções:}
\end{itemize}

\begin{itemize}[<+->]
\tightlist
\item
  \alert<2>{§ 4º Sem prejuízo da responsabilidade civil, as infrações previstas 
  neste artigo serão punidas com as seguintes sanções:}
\end{itemize}

\begin{itemize}[<+->]
\tightlist
\item
  \alert<3>{I - embargo de obra, serviço ou atividade, até a manifestação da 
  União quanto à regularidade de ocupação;}
\end{itemize}

\begin{itemize}[<+->]
\tightlist
\item
  \alert<4>{II - aplicação de multa;}
\end{itemize}

\begin{itemize}[<+->]
\tightlist
\item
  \alert<5>{III - desocupação do imóvel; e}
\end{itemize}

\begin{itemize}[<+->]
\tightlist
\item
  \alert<6>{IV - demolição e/ou remoção do aterro, construção, obra, cercas ou 
  demais benfeitorias, bem como dos equipamentos instalados, à conta de quem os 
  houver efetuado, caso não sejam passíveis de regularização.}
\end{itemize}

\end{frame}

\hypertarget{competuxeancia}{%
\section{Competência}\label{competuxeancia}}

\begin{frame}{Competência}

\begin{itemize}[<+->]
\tightlist
\item
  \alert<1>{A competência para a Autorização de Obras em imóveis da União foi
  delegada aos Superintendentes Estaduais do Patrimônio da União pela Portaria
  83\footcite{portaria-83}:}
\end{itemize}

\begin{itemize}[<+->]
\tightlist
\item
  \alert<2>{\emph{Art. 15. Fica subdelegada competência aos Superintendentes do
  Patrimônio da União, observadas as disposições legais e regulamentares, para
  autorizar, mediante as condições constantes do Anexo I:}}
\end{itemize}

\begin{itemize}[<+->]
\tightlist
\item
  \alert<3>{\emph{(...)}}
\end{itemize}

\begin{itemize}[<+->]
\tightlist
\item
  \alert<4>{\emph{VI - autorização de obra em áreas de uso comum do povo de
  domínio da União, quando a intervenção a ser realizada não alterar essa
  característica e for dispensada posterior cessão;}}
\end{itemize}

\begin{itemize}[<+->]
\tightlist
\item
  \alert<5>{Ver também anexo X da Portaria 11\footcite{portaria-11}, de 31 de Janeiro de 2018}
\end{itemize}

\end{frame}

\hypertarget{anuxe1lise-tuxe9cnica}{%
\section{Análise Técnica}\label{anuxe1lise-tuxe9cnica}}

\begin{frame}{Análise Técnica}

\begin{itemize}[<+- | alert@+>]
  \item a) Nome do requerente.
  \item b) Em que consiste a Obra (Apresentação e Descrição).
  \item c) Finalidade da Obra.
  \item d) Tamanho da área da Obra em $\mathrm{m}^2$
  \item e) Endereço completo do local da Obra. 
  \item f) Se a Obra acresce ou diminui terreno da União ou se faz movimentação de terra ou areia.
  \item g) Nome e telefone do responsável pelas informações da Obra junto à SPU/SC. 
  \item h) Informar a previsão para inicio e conclusão das obras.
\end{itemize}

\end{frame}

\begin{frame}{Documentação}
\protect\hypertarget{documentauxe7uxe3o}{}

\begin{itemize}[<+- | alert@+>]
  \item Projeto Básico da Obra e Memorial da área abrangida.
  \item Demonstrativo de Viabilidade Econômica;
  \item Parecer da Capitania dos Portos;
  \item Licença Ambiental;
  \item Relatório Fotográfico;
  \item Memorial Descritivo sobre benfeitorias;
  \item Declaração de Comunidades Tradicionais;
  \item Declaração de Unidades de Conservação;
  \item Declaração de Responsabilidade pela Manutenção.
\end{itemize}

\end{frame}

\hypertarget{exemplo}{%
\section{Exemplo}\label{exemplo}}

\begin{frame}{Exemplo}

\begin{itemize}[<+->]
\tightlist
\item
  \alert<1>{10154.110758/2019-73, Portaria 13.399, de 02 de Junho de 2020}
\item
  \alert<2>{\emph{\textbf{O SUPERINTENDENTE DO PATRIMÔNIO DA UNIÃO EM SANTA
  CATARINA, DO MINISTÉRIO DA ECONOMIA}, no uso da competência que lhe foi
  subdelegada pelo art. 15, inc. VI, da Portaria nº 83, de 28 de agosto de 2019,
  c/c o art. 68 Anexo X, da Portaria nº 11, de 31 de janeiro de 2018 - MPDG, e
  tendo em vista o disposto no art. 6º, do Decreto-Lei nº 2.398, de 21 de dezembro
  de 1987, com a nova redação que lhe foi conferida pela Lei nº 13.139, de 26 de
  junho de 2015, e de acordo com os elementos que integram o Processo nº
  10154.110758/2019-73,}}
\end{itemize}

\begin{itemize}[<+->]
\tightlist
\item
  \alert<3>{\emph{\textbf{RESOLVE:}}}
\end{itemize}

\begin{itemize}[<+->]
\tightlist
\item
  \alert<4>{\emph{Art. 1º - Autorizar a interessada, Interligação Elétrica
  Biguaçu S.A. (CNPJ: 28.218.051/0001-03), a realizar a execução de obras
  referentes à instalação de linhas de transmissão sub-aquáticas entre Biguaçu e
  Florianópolis (baía norte), em águas da União, na forma dos elementos constantes
  do processo nº 10154.110758/2019-73;}}
\end{itemize}

\begin{itemize}[<+->]
\tightlist
\item
  \alert<5>{\emph{(...)}}
\end{itemize}

\end{frame}

\begin{frame}{Conclusão}
\protect\hypertarget{conclusuxe3o}{}

\Huge\center \textbf{\textsc{Obrigado!}}

\end{frame}


  \begin{frame}[allowframebreaks]{}
  \bibliographytrue
  \printbibliography[heading=none]
  \end{frame}


\end{document}
